\documentclass{article}
\title{1st ODE Integrating Factor}
\author{Vivian Duong}
\usepackage{amssymb}
\usepackage{amsmath}
\usepackage{amsthm}
\usepackage[pdftex]{hyperref}
\newtheorem{definition}{Definition}

\begin{document}

\maketitle

Claim: Suppose the First Order Differential Equation
\begin{equation} \label{eqref:1}
\frac{dy}{dt} + p(t)y = g
\end{equation}
where $p(t)$ and $g(t)$ are $\textbf{continuous}$ functions. Continuous functions have no holes or breaks in it. If you were to draw it, you would never need to pick up your pencil.

\begin{definition} [continuous function]
A function f(x) is continuous at x=a if
$\lim_{x\rightarrow a} f(x) = f(a)$
A function is continuous on the interval [a,b] if it is continuous at each point in the interval.
\end{definition}

And suppose there exists an integrating factor, $\mu(t)$ where
\begin{equation}
\mu(t) = e^{\int p(t)\mathrm{d}t} 
\end{equation}
Then the solution of $\eqref{eqref:1}$ is
\begin{equation}
y = \frac{\int \mu(t)g(t)\mathrm{d}t + c }{\mu(t)}
\end{equation}

\begin{proof}
This proof can also function as step-by-step instructions on how to solve a First Order Differential Equation.

\begin{enumerate}
\item Multiply both sides of $\ref{eqref:1}$ by $\mu$

$\mu(t)\frac{dy}{dt} + \mu(t)p(t)y = \mu(t)g$

$\mu(t)p(t) = e^{\int p(t)\mathrm{d}t}p(t)
 = \mu(t)'$

$\mu(t)y' + \mu'(t)y = \mu(t)g$

$(\mu y)' = \mu g$

$\int(\mu y)' \mathrm{d}t = \int\mu g \mathrm{d}t$

$\mu y + c = \int\mu g \mathrm{d}t$

$y(t) =\dfrac{\int\mu(t) g(t) \mathrm{d}t + c}{\mu(t)}$

\end{enumerate}
\end{proof}
\end{document}