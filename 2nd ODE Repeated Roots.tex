\documentclass{article}
\title{2nd ODE Repeated Roots}
\author{Vivian Duong}
\usepackage{amssymb}
\usepackage{amsmath}
\usepackage{fullpage}
\usepackage{amsthm}
\usepackage[pdftex]{hyperref}
\newtheorem{definition}{Definition}

\begin{document}
\maketitle
Prove that the 2nd root of a characteristic equation of repeated roots is $te^{r_2t}$. Suppose that $r_1$ and $r_2$ are roots of $ar^2 + br + c = 0$ and that $r_1 \neq r_2$; then $e^{r_1t}$ and $e^{r_2t}$ are solutions of the differential equation $ay" + by' + cy = 0$. Show that $\phi(t; r_1, r_2) = \frac{e^{r_2t}-e^{r_1t}}{r_2-r_1}$ is also a solution for $r_2 \neq r_1$. Then think of $r_1$ as fixed and use $L'H\hat{o}pital's Rule$ to evaluate the limit of $\phi(t:r_1, r_2)$ as $r_2\rightarrow r_1$, thereby obtaining the 2nd solution in the case of equal roots.

Suppose the 2nd Order Differential Equation (ODE)
\begin{equation} \label{eqref:1}
ay" + by' + c =0
\end{equation}
and $\phi$ is a solution.

Then its characteristic equation is
\begin{equation} \label{eqref:2}
ar" + br' + c = o
\end{equation}

Now suppose
\begin{equation}
r_1 \neq r_2
\end{equation}

Therefore,

\begin{equation}
\phi_1 = e^{r_1t}
\end{equation}
and
\begin{equation}
\phi_2 = e^{r_2t}
\end{equation}
are solutions of $\eqref{eqref:1}$

To show that 
\begin{equation} \label{eqref:6}
\phi = \dfrac{e^{r_2t}-e^{r_1t}}{r_2-r_1}
\end{equation}
is a solution, plug in $\phi$:
\begin{equation}
a\phi" + b\phi' +c\phi = 0
\end{equation}
We know $\phi$ from $\eqref{eqref:6}$

\begin{equation}
\phi' = \dfrac{r_2e^{r_2t}-r1e^{r_1t}}{r_2-r_1}
\end{equation}

\begin{equation}
\phi" = \dfrac{r_2^2e^{r_2t}-r_1^2e^{r_1t}}{r_2-r_1}
\end{equation}

Actually plugging in $\phi$
\begin{center}
$a(\dfrac{r_2^2e^{r_2t}-r_1^2e^{r_1t}}{r_2-r_1}) + b(\dfrac{r_2^2e^{r_2t}-r_1^2e^{r_1t}}{r_2-r_1}) + c(\dfrac{r_2^2e^{r_2t}-r_1^2e^{r_1t}}{r_2-r_1}) = 0$

$(\dfrac{1}{r_2-r_1})[a(r_2^2e^{r_2t}-r_1^2e^{r_1t}) + b(r_2e^{r_2t}-r1e^{r_1t}) + c(e^{r_2t}-e^{r_1t}) = 0$

$(\dfrac{1}{r_2-r_1})[e^{r_1t}(ar_1^2 +br_1 + c) - e^{r_2t}(ar_2^2 + br_2 + c)]=0$
\end{center}

From $\eqref{eqref:2}$, we know that

\begin{center}
$(\dfrac{1}{r_2-r_1})[e^{r_1t}(0) - e^{r_2t}(0)]=0$

$(\dfrac{1}{r_2-r_1})[0]=0$

$0 = 0$
\end{center}

Therefore, we know for sure that $\phi$ is a solution of $\eqref{eqref:1}$

\begin{center}
$ \lim_{r_2 \rightarrow r_1} \phi $

$  \lim_{r_2 \rightarrow r_1} \dfrac{e^{r_2t}-e^{r_1t}}{r_2-r_1} $
\end{center}

Using $L'h/hat{o}pital's Rule$ by integrating $\phi$ with respect to $r_2$

\begin{center}
$  \lim_{r_2 \rightarrow r_1} \dfrac{te^{r_2t}}{1} $


$te^{r_1t}$
\end{center}
Therefore, when $r_1 = r_2$, the second solution of $\eqref{eqref:1}, \phi_2 = te^{rt}$

\newpage

Prove that the second solution for a characteristic equation, ay" + by' + cy = 0, with repeated roots is $cte^{at}$.

One way to prove this is with using two pieces of information:

\begin{enumerate}
\item The characteristic equation for repeated roots is of the form 
\begin{equation} \label{eqref:10}
y" + 2ay' + a^2y = 0
\end{equation}
\item \begin{definition} [Abel's Formula]
Suppose $ y" + p(t)y' + q(t)y = 0$
\begin{equation}
Wronskian(y_1, y_2) = ce^{-\int p(t) \mathrm{d}t} 
\end{equation}
\begin{center}
det$
\begin{vmatrix}
y_1 & y_2\\
y_1' & y_2'
\end{vmatrix}
= ce^{-\int p(t) \mathrm{d}t}
$
\end{center}
\end{definition}
\end{enumerate}

To show that $\eqref{eqref:10}$ is a characteristic equation with repeated roots.
\begin{center}
$y" + 2ay' + a^2y = 0$
$r^2 + 2ar + a^2 = 0$
$(r + a)^2 = 0$
$r_1 = r_2 = -a$
\end{center}
Therefore, the Wronskian of the characteristic equation with repeated roots $\eqref{eqref:10}$ is $ce^{-int 2a \mathrm{d}t}$.

\begin{center}
$W(y_1,y_2) = ce^{-\int 2a \mathrm{d}t}$

$W(y_1,y_2) = ce^{-2at}$

$W(y_1,y_2) = ce^{-2at}$


det$\begin{vmatrix}
y_1 & y_2\\
y_1' & y_2'
\end{vmatrix}= ce^{-2at}
\newline
\begin{vmatrix}
e^{-at} & y_2\\
-ae^{-at} & y_2'
\end{vmatrix}
= ce^{-2at}
$
\newline

$y_2'e^{-at} - ae^{at}y_2 = ce^{-2at}$

$(e^{-at}y_2)' = ce^{-2at}$

$e^{-at}y_2 = c_1te^{-2at} + c_2e^{-2at}$

$y_2 = c_1te^{-at} + c_2e^{-at} $

$c_2e^{-at}$ absorbs into $y_1 = ce^{-at}$

$y_2 = cte^{-at}$
\end{center}
\end{document}
