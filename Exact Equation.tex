\documentclass{article}
\title{Exact Equation}
\author{Vivian Duong}
\date{}
\usepackage{enumerate}
\usepackage{amssymb}
\usepackage{amsmath}
\usepackage{fullpage}
\usepackage{amsthm}
\usepackage[pdftex]{hyperref}
\usepackage{soul}
\usepackage{ulem}
\newtheorem{definition}{Definition}
\newtheorem{theorem}{Theorem}[section]

\begin{document}
\begin{proof}
\begin{theorem}
Given $M(x,y) + N(x,y)\dfrac{\mathrm{d}y}{\mathrm{d}x} = 0$, $M_y=N_x$, and $\psi (x,y(x)) = \int M\mathrm{d}y = \int N\mathrm{d}x$, $\psi(x,y) = c$, which is the implicit solution.
\end{theorem}

\begin{center}
$\psi_x = M$

$\psi_y = N$

$\psi_x + \psi_y\dfrac{\mathrm{d}y}{\mathrm{d}x} = 0$


$\psi_x\mathrm{d}x + \psi_y\mathrm{d}y = 0$


$\dfrac{\partial \psi}{\partial x}\dfrac{\mathrm{d}x}{\mathrm{d}x} + \dfrac{\partial \psi}{\partial y}\dfrac{\mathrm{d}y}{\mathrm{d}x} =0$

$\int \dfrac{\mathrm{d}}{\mathrm{d}x} \psi(x,y(x)) = \int 0$
\end{center}
\begin{flushright}
\begin{definition}[Partial Derivatives Chain Rule]


$\dfrac{\mathrm{d}}{\mathrm{d}t}(z(x_1,...,x_n)) = \dfrac{\partial z}{\partial x_1}\dfrac{\partial x_1}{\partial t} + \dfrac{\partial z}{\partial x_2}\dfrac{\partial x_2}{\partial t} + ... + \dfrac{\partial z}{\partial x_n}\dfrac{\partial x_n}{\partial t}$

\end{definition}
\end{flushright}
\begin{center}
$\psi (x,y) = C$
\end{center}

\end{proof}
\end{document}