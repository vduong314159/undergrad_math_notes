\documentclass{article}
\title{Complex Roots of the Characteristic Equation}
\author{Vivian Duong}
\date{}
\usepackage{enumerate}
\usepackage{amssymb}
\usepackage{amsmath}
\usepackage{fullpage}
\usepackage{amsthm}
\usepackage[pdftex]{hyperref}
\usepackage{soul}
\usepackage{ulem}
\newtheorem{definition}{Definition}
\newtheorem{theorem}{Theorem}[section]

\begin{document}
The gamma function is denoted by $\Gamma(p$) and is defined by the integral

\begin{equation}
\Gamma(p+1) = \int_o^\infty e^{-x}x^p\mathrm{d}x
\end{equation}

The integral converges as $x \rightarrow \infty$ for all $p$. For $p < 0$ it is also improper because the integrand becomes unbounded as $x \rightarrow 0$. However, the integral can be shown to converge at $x=0$ for $p>-1$.

I Show that , for $p>0$,

\begin{equation}\label{eqref:1}
\Gamma(p+1) = p\Gamma(p)
\end{equation}

To begin,
\begin{center}
$\Gamma(p+1) = \int_0^\infty e^{-x}x^p\mathrm{d}x$
\end{center}

\begin{flushright}
$\int u \mathrm{d}v = uv - \int v \mathrm{d}u$

$u = x^p, v = e^{-x}$

$du = px^{p-1}, dv = -e^{-x}$
\end{flushright}

\begin{center}
$ = \lim_{A \rightarrow \infty} [\dfrac{x^p}{e^x}|_0^A] + p \int_0^\infty e^xx^{p-1} \mathrm{d}x$

$ = \lim_{A \rightarrow \infty} \dfrac{A^p}{e^A} - \dfrac{0^p}{e^0} + p \int_0^\infty e^xx^{p-1} \mathrm{d}x$

$ =  p \int_0^\infty e^xx^{p-1} \mathrm{d}x$

$ = p\Gamma (p)$
\end{center}

II Show that $\Gamma(1) = 1$
\begin{center}
$\Gamma(p) = \int_0^\infty e^xx^{p-1} \mathrm{d}x$

$\Gamma(1) = \int_0^\infty e^xx^{0} \mathrm{d}x$

$\Gamma(1) = \int_0^\infty e^x \mathrm{d}x$

$\Gamma(1) = lim_{A \rightarrow \infty} [-e^{-x}|_0^A]$

$\Gamma(1) = 0 -(-1)$

$\Gamma(1) = 1$
\end{center}

III Show that if $p$ is a positive integer n,
\begin{center}
$\Gamma(n+1) = n!$

$\Gamma(p+1) = p\Gamma (p)$

$\Gamma(n+1) = n\Gamma (n)$

$ = n(n-1)\Gamma (n-1)$

$ = n(n-1)(n-2)\Gamma (n-2)$

$ = n(n-1)(n-2)...\Gamma (1)$

$ = n(n-1)(n-2)...3.2.1$
\end{center}
\begin{flushright}
$\Gamma(1) = 1$
\end{flushright}
\begin{center}
$ \Gamma (n+1)= n!$
\end{center}

IV Show that, for p>0, 
\begin{equation}
p(p+1)(p+2)...(p+n-1) = \dfrac{\Gamma (p+n)}{\Gamma (p)}
\end{equation}

\begin{flushright}
Recall $\eqref{eqref:1}$
$\Gamma(p+1) = p\Gamma(p)$ for $p>0$
\end{flushright}
\begin{center}
Let $\Gamma (p+n) = p\Gamma (p+n)$ for $(p+n)>0$

$ = (p+n-1)\Gamma (p+n-1)$

$ = (p+n-2) \Gamma (p+n-2)$

.

:

$\Gamma (p+n) = (p+n-1)(p+n-2)...(p+2)(p+1)p\Gamma (p)$

$\dfrac{\Gamma (p+n)}{\Gamma (p)} = (p+n-1)(p+n-2)...(p+2)(p+1)p$


\end{center}
\end{document}