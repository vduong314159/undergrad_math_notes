\documentclass{article}
\title{Laplace Transform}
\author{Vivian Duong}
\date{}
\usepackage{enumerate}
\usepackage{amssymb}
\usepackage{amsmath}
\usepackage{fullpage}
\usepackage{amsthm}
\usepackage[pdftex]{hyperref}
\usepackage{soul}
\usepackage{ulem}
\newtheorem{definition}{Definition}
\newtheorem{theorem}{Theorem}[section]

\begin{document}
I Show that $\mathcal{L} f(ct) = \dfrac{1}{c} F(\dfrac{s}{c})$

\begin{flushright}
\begin{definition}[Laplace Transform]
$F(s) = \mathcal{L}f(t), s>a\geq 0$
$s>ca$
\end{definition}
\end{flushright}

\begin{center}

$\mathcal{L} f(ct) = \int_0^\infty e^{-st} f(ct) \mathrm{d}t$
\end{center}

\begin{flushright}
$u = ct,   \dfrac{u}{c} = t$

$\dfrac{\mathrm{d}u}{\mathrm{d}t} = c$

$\dfrac{1}{c} \mathrm{d}u = \mathrm{d}t$
\end{flushright}
\begin{center}
$\mathcal{L} f(ct) = \dfrac{1}{c} \int_0^\infty e^{-\dfrac{su}{c}}f(u)\mathrm{d}u$

\begin{flushright}
$t = u$

$dt = du$
\end{flushright}

$\mathcal{L} f(ct) = \dfrac{1}{c} F(\dfrac{s}{c})$
\end{center}

\end{document}