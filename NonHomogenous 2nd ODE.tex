\documentclass{article}
\title{Non-Homogenous 2nd ODE}
\author{Vivian Duong}
\usepackage{amssymb}
\usepackage{amsmath}
\usepackage{amsthm}
\usepackage[pdftex]{hyperref}
\newtheorem{definition}{Definition}

\begin{document}
Claim: If $y = \phi(t)$ is a solution of the non-homogenous 2nd order differential equation 

\begin{equation} \label{eqref:1}
y" + p(t)y' + q(t)y = g(t)
\end{equation}
where $g(t)$ is \textit{not} always zero, then 
\begin{equation} \label{eqref:2}
y = c\phi(t)
\end{equation}
is not a solution.

\begin{proof}
Because $\eqref{eqref:2}$ is a solution of $\eqref{eqref:1}$, then

\begin{equation} \label{eqref:3}
\phi"(t) + p(t)\phi'(t) + q(t)\phi(t) = g(t)
\end{equation}

Deriving $\eqref{eqref:2}$

\begin{center}
$y = c\phi(t)$

$y' = c\phi'(t)$

$y" = c\phi"(t)$

\end{center}
Plugging it into $\eqref{eqref:1}$


\begin{equation}
c\phi"(t) + p(t)c\phi'(t) + q(t)c\phi(t) \neq g(t)
\end{equation}
\begin{center}
$c(\phi"(t) + p(t)\phi'(t) + q(t)phi(t)) \neq g(t)$

$c(g(t)) \neq g(t)$
\end{center}

\end{proof}
\end{document}