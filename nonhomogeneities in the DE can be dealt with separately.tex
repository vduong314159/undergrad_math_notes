\documentclass{article}
\title{A soln of a characteristic eqn is either zero everywhere or else can be zero at most once if its roots are real}
\author{Vivian Duong}
\date{}
\usepackage{enumerate}
\usepackage{amssymb}
\usepackage{amsmath}
\usepackage{fullpage}
\usepackage{setspace} %line spacing
\usepackage{amsthm}
\usepackage[pdftex]{hyperref}
\usepackage{soul}
\usepackage{ulem} %to strike out text \sout
\newtheorem{definition}{Definition}
\newtheorem{theorem}{Theorem}[section]

\begin{document}
Show that the solution of the initial value problem 

\begin{equation}
L[y] = y" + p(t)y' + q(t)y = g(t), y(t_0)=y_0,  y'(t_0)=y_0' 
\end{equation}
can be written as $y = u(t) + v(t)$, where $u$ and $v$ are solutions of the two initial value problems

\end{document}